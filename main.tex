\def\homeworkname{Star Network}
\documentclass[assignment = 12]{homework}

\usepackage{caption, subcaption, pdfpages, float}
\usepackage{graphics, wrapfig, pgf, graphicx}
\usepackage{enumitem}


% pacotes para importar código
\usepackage{caption, booktabs}
\usepackage[inkscapepath=build/inkscape]{svg}
\usepackage[section, newfloat, outputdir=build/]{minted}
\definecolor{sepia}{RGB}{252,246,226}
\setminted{
    bgcolor = sepia,
    style   = pastie,
    frame   = leftline,
    autogobble,
    samepage,
    python3,
    breaklines
}
\setmintedinline{
    bgcolor={}
}

% ambientes de códigos de Python
\newmintedfile[pyinclude]{python3}{}
\newmintinline[pyline]{python3}{}
\newcommand{\pyref}[2]{\href{#1}{\texttt{#2}}}

% \SetupFloatingEnvironment{listing}{name=Código}
% \captionsetup[listing]{position=below,skip=-1pt}

\usepackage{csquotes}
\usepackage[
    style    = verbose-ibid,
    autocite = footnote,
    notetype = foot+end,
    backend  = biber
]{biblatex}
\addbibresource{references.bib}
\usepackage[section]{placeins}

\usepackage[hidelinks]{hyperref}
\usepackage[noabbrev, nameinlink]{cleveref}
\hypersetup{
    pdftitle  = {MO412/MC908 - RA187679},
    pdfauthor = {Tiago de Paula}
}

\newcommand{\textref}[2]{
    \hyperref[#2]{#1 \ref*{#2}}
}

\renewcommand{\vec}[1]{\mathbf{#1}}

\DeclareMathOperator{\round}{round}

\usepackage{import, multirow}
\usepackage{pgf, tikz}
\usetikzlibrary{matrix}
\usetikzlibrary{positioning}
\usetikzlibrary{automata}
\usetikzlibrary{shapes}

\usepackage{wrapfig}
\usepackage{booktabs}
\input{simbolos}

\newenvironment{kmatrix}[1][1.3cm]{
    \begin{tikzpicture}[node distance=0cm]
        \tikzset{square matrix/.style={
                matrix of nodes,
                column sep=-\pgflinewidth, row sep=-\pgflinewidth,
                nodes={draw,
                    minimum height=#1,
                    anchor=center,
                    text width=#1,
                    align=center,
                    inner sep=0pt
                },
            },
            square matrix/.default=#1
        }
}{
    \end{tikzpicture}%
}

\newcommand*{\Scale}[2][4]{\scalebox{#1}{\ensuremath{#2}}}%

\newcommand{\red}[1]{\textcolor{red}{\textbf{#1}}}
\def\qm{?}

\begin{document}
    \pagestyle{main}

    Consider a star network, where a single node is connected to $N-1$ degree-one nodes. Assume that $N$ is much larger than 1. Your goal is to compute the degree correlation coefficient of this network as a function of $N$, using Formulas \hyperref[eq:coefficient]{7.11} and \hyperref[eq:variance]{7.12} in \href{http://networksciencebook.com/chapter/7#measuring-degree}{Box 7.2 of the adopted book}, following the steps:

    \begin{enumerate}[label={(\alph*)}]
        \item compute the numerator of 7.11; \label{item:a}
        \item compute the denominator of 7.11 using 7.12; \label{item:b}
        \item divide the result of \ref{item:a} by the result of \ref{item:b}. \label{item:c}
    \end{enumerate}

    \section{The Network}

\begin{wrapfigure}{r}{0.3\textwidth}
    \centering
    \includesvg[width=0.25\textwidth]{images/star.svg}

    \caption{Star network with $N = 13$ nodes.}
    \label{fig:example}
\end{wrapfigure}

Looking at \cref{fig:example}, we can clearly that every link is connected to a node of degree 1 and the central node, with degree $N-1$. Therefore, $e_{1,N-1} = 1$ when $N > 1$ and $e_{i,j} = 0$ for any other distinct pair $i,j$.

From this, we get \[
    q_1 = \sum_{j = 1}^N e_{1,j} = e_{1,N-1} = 1
\]

Similarly, we reach $q_{N-1} = e_{1,N-1} = 1$. For $i \ne 1$ and $i \ne N-1$, however, $q_i = 0$.

    \section{Degree Correlation Coefficient}

The Degree Correlation Coefficient $r$ is equivalent to the Pearson correlation coefficient for network degrees and is given by
\[
    r = \sum_{j k} \frac{j k \left(e_{j,k} - q_j q_k\right)}{\sigma^2} = \frac{1}{\sigma^2} \sum_{j k} j k \left(e_{j,k} - q_j q_k\right)
    \tag{7.11} \label{eq:coefficient}
\]

Where $\sigma^2$ is comparable to the variance and is given by
\[
    \sigma^2 = \sum_{k} k^2 q_k - \left(\sum_{k} k q_k\right)^2
    \tag{7.12} \label{eq:variance}
\]

\subsection{Numerator}

    The numerator, here represented as $C$, is
    \[
        C = \sigma^2 r = \sum_{j k} j k \left(e_{j,k} - q_j q_k\right)
    \]

\subsection{Denominator}

\subsection{Result}

    Finally, we can compute the degree correlation coefficient as
    \[
        r = \frac{?}{\sigma^2} = ???
    \]


\end{document}
